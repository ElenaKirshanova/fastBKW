\section{Practical Hardness of \LWE} \label{sec:LWEasBDDPr}

In this section we leave asymptotics and draw our attention to the practical hardness of the Learning with Errors problem. 

The results of the previous section tell us that the two-phase $\BKZ+\ENUM$ approach to solve \LWE in $\poly(n)$-memory regime performs better than $\DUAL$ or Embedding attacks when only $\TLandau(n)$ samples are provided. Moreover, it is only slightly worse than the $\DUAL$ algorithm when the latter can access exponentially many samples. So in the most realistic scenario -- $\poly(n)$ memory, limited number of samples -- \BKZ reduction followed by enumeration is the right strategy.

We already saw in Thm.~\ref{thm:BalanceSuperExp} that in the two-phase algorithm $\BKZ+\ENUM$, both steps have complexities of order $2^{\TLandau(n \log n)}$ since the \BKZ parameter $\beta$ optimizes the attack when $\beta = \TLandau(n)$, where we made the constant for $\beta$ explicit. 

These arguments are well-suited to conclude on the asymptotics. On the practical side, however, the $\BKZ$ algorithm is notoriously hard to implement and until very recently\footnote{On the 22.09.16, Albrecht et al. announced \cite{fplll} the release of the \BKZ 2.0 algorithm, which asymptotically meets the desired $2^{\bigO(n \log n)}$ complexity.}, the only available implementation of lattice-basis reduction was provided in Shoup's NTL library \cite{Sho} and most of the complexity benchmarks (\cite{APS15,MicReg09,EC:NguReg06}) were obtained by running this implementation. During the execution, the \BKZ algorithm in NTL calls Fincke-Pohst enumeration \cite{FinPoh83} as an \SVP solver. The running time of this enumeration procedure is of order $2^{\bigO(\beta^2)}$, thus resulting in much worse complexity for the reduction than theory suggests.

So it is reasonable to try to shift the workload of our \BDD attack from the reduction to the enumeration phase. This approach is even more advantageous once we notice that the enumeration algorithm -- a tree-traversal routine -- is amenable to efficient parallelization. 

In this section, we present the real running times of the \BKZ+\ENUM attack on \LWE with our parallelized implementation of the enumeration step.\footnote{The code is available on-line: \url{https://github.com/pfasante/cvp-enum}} The experiments are carried out in combination with the \BKZ algorithm from the NTL library. We note that, to perform the attack, one can use any other \BKZ implementation to preprocess a basis and then run our enumeration algorithm.

From our experimental results we draw two main conclusions: (1) the \BDD enumeration algorithms described in Sect.~\ref{sec:LWEasBDDAs} can be almost perfectly parallelized by splitting the enumeration tree into sub-tress and traversing the sub-trees in parallel; (2) the combinatorial $\BKW$-type attacks (\cite{C:GuoJohSta15, C:KirFou15}) are \emph{not} better in practice than the lattice-based attacks even for parameters favorable for the former (e.g.\ small or even binary secret).  We emphasize on \emph{practical} superiority of the lattice-based methods over combinatorial despite the fact that asymptotics might present a different picture (cf.\ Table~\ref{table:compareTable}, Fig.~\ref{fig:LWEPlots}). %Hence, in order to estimate practical hardness of concrete \LWE instances, one should focus on lattice-based attacks. 

The roadmap of this subsection is as follows. First, we describe a single-threaded tree-traversal enumeration algorithm. Next, we show how to distribute the traversal of sub-trees among several threads to execute it in parallel. We discuss certain tweaks of the \BDD attack one can apply to variants of \LWE. At the end, we present complexities of real-time attacks on concrete \LWE instances (see Table~\ref{tabel:RunTimesLWE}).

\subsection{Single threaded implementation} \label{sec:SingleThread}

Here we give an alternative representation of the pruning algorithm $\GenPrun$ (Alg.~\ref{alg:GenPrun}) suitable for efficient implementation.
Recall that a \BDD enumeration algorithm for \LWE with parameters $(n, \alpha, q)$ receives as input a $\beta$-reduced lattice-basis $\BMat \in \Z_q^{m \times m}$ and a target $\tvec \in \Z_q^m$ with a promise that $\tvec$ is only $\TLandau(\alpha q \sqrt{m})$ away from a vector $\vvec \in \qLat(\BMat)$ we search for. In addition, the algorithm is provided with a description of a bounding function $\B$ which is used to prune the enumeration tree (see examples of $\B$ in Sect.~\ref{sec:GenPrun}).

Algorithm $\GenPrunDepth$ (Alg.~\ref{alg:GenPrunDepth}) is a \emph{depth-first} description of Alg.~\ref{alg:GenPrun} from the previous section. It constructs an enumeration tree where a $k$-level node stores (1) a target vector $\tvec^{(k)}$, (2) a coefficient-vector $\cvec$ of a candidate-solution $\xvec^{(k)} = \sum_{i=k+1}^m \cvec^{(k)}_i \bvec_i$ ($\cvec$ is constructed starting with its $m\th$ coordinate $c_m$ down to $c_1$), and (3) an accumulated \emph{error-length} $e^{(k)} = \sum_{i=k+1}^{m} e'^2_i \|\wbvec_i \|^2$, where $\evec^{(k)} = \sum_{i=k+1}^{m} e'_i \wbvec_i$ is the error accumulated by a node on level $k$. On the root we have $k=m, e^{(m)}=0, t^{(m)} = \tvec$. The leaves ($k=0$) give candidate-solutions $\xvec = \sum_{i=1}^{m} \cvec_i \bvec_i $ with error-length $e^{(0)} =\| \tvec - \xvec \|$. Different paths have different coefficient-vectors $\cvec$. Depth-first traversing is memory-efficient (as opposed to the recursive version given in Alg.~\ref{alg:GenPrun}) since we consider only one path $\cvec$ at a time and decide whether the corresponding error is smaller than the previously found or not. 

Note that instead of keeping the coordinates of a partial error-vector as in Alg.~\ref{alg:GenPrun}, we store only its length. We do so by observing that for bounding functions $\B$ of our interest (like Length Pruning), we only need the error-\emph{length} but not its individual coordinates to evaluate $\B$. So for the algorithm $\GenPrunDepth$ we simplify the definition of a bounding function and consider only functions $\B: \Q_{\geq 0} \rightarrow \Q_{\geq 0}$ that take a squared error-length as input and output the remaining allowed length. From the value $\B(e^{(k)})$, we compute the number of children for a node with the (squared) error-length $e^{(k)}$ (line \ref{algline:GenPrunDNumChildren}), and all the relevant information for its left-most child (lines 11--13). From this left-most child we go down-left again. Once a leaf is reached, we compare its error-length $e^{(0)}$ with the error $\text{minLen}$ of the best (i.e.\ the shortest) solution found so far. In case $e^{(0)}$ is smaller than $\text{minLen}$, a new candidate-solution is constructed from the coefficient vector $\cvec$ of the current path (line \ref{algline:GenPrunDSol}). At the end, the returned solution has the minimal error-length among all the solutions considered by the algorithm. 

%
% GenPruning Depth-first
%

\begin{algorithm}[t]
\caption{$\GenPrunDepth (\BMat, \protect \tvec, \B^{(k)})$}
\label{alg:GenPrunDepth}
\textbf{Input:} $\BMat=(\bvec_1, \ldots, \bvec_m) \in \Z^{m \times m}, \tvec \in \Z^m$, a family of bounding functions $\B^{(k)}: \Q \rightarrow \Q$ \\
\textbf{Output:} $\xvec\in \Lat(\BMat)$ close to $\tvec$ and $e = \|\evec\| = \| \tvec - \vvec \| $
%\vspace{8pt}
\begin{algorithmic}[1]
\State $\tvec^{(m)}\gets \tvec, e^{(m)} \gets 0, k \gets m $.
\State Let $\widetilde{\BMat}\gets\GSO(\BMat)$
\If {$m=0$} \Return $(\tvec^{(m)}, e)$
\EndIf
\State $(\tvec^{(0)}, \text{minLen}) \gets \NP (\BMat, \tvec)$
\While{(true)}
	\If{$(k>0)$}
		\State $Int \gets \sqrt{ \B^{(k)}(e^{(k)}) } / \| \wbvec_k \|$  \Comment Number of children \label{algline:GenPrunDNumChildren}
		\State $c^{*} \gets \langle \tvec^{(k)}, \wbvec_k  \rangle / \normalabs{\wbvec_{k}}^2 $
		\State $c_{\text{min}} \gets \lceil c^{*} - \tfrac{1}{2} Int \rceil$ \Comment Left-most child
		\State $c_{\text{max}} \gets \lfloor c^{*} + \tfrac{1}{2} Int \rfloor$ \Comment Right-most child
		\State $\cvec_k \gets c_{\text{min}}$
		\State $\tvec^{(k-1)} \gets \tvec^{(k)} - \cvec_k \bvec_k $	 \Comment Project onto $U^{(k)} = \cvec_k \wbvec_{k}+ \Span(\bvec_1, \ldots, \bvec_{k-1})$
		\State $e^{(k-1)} \gets e^{(k)} + (\cvec_k - c^{*})^2 \|\wbvec_{k}\|^2$ \Comment Compute the squared error-length
		\State $k \gets k-1$ \Comment Go down the tree
	\Else \Comment On a leaf
		\If{($e^{(k)} < \text{minLen}$)}
			\State $\xvec \gets \sum_{i=1}^k c^{(i)} \bvec_i$ \Comment Current best solution \label{algline:GenPrunDSol}
		\EndIf
		\Repeat \Comment Traverse up
			\If{($k=0 \text{ AND } \cvec_k > c_{\text{max}}$)} \Comment On the root, no right siblings
				\State \Return $(\xvec, \text{minLen})$
			\EndIf
			\State $k \gets k+1$
		\Until{($\cvec_k \geq c_{\text{max}}$)}

		\State $\cvec_k \gets c^{(k)} + 1$ \Comment Traverse to the right sibling
		\State $\tvec^{(k-1)} \gets \tvec^{(k)} - \lceil \cvec_k \rfloor \bvec_k $
		\State $e^{(k-1)} \gets e^{(k)} + (\cvec_k - \lceil \cvec_k \rfloor)^2 \|\wbvec_{k}\|^2$

	\EndIf
\EndWhile

\State \Return $(\tvec^{(0)}, e^{(0)})$

\end{algorithmic}
\end{algorithm}

The algorithm described above traverses the enumeration tree in (depth-first) \emph{left-most} child manner (on line 11, we start with $c_{\text{min}}$ that represents the left-most child). This `classical' traversal is depicted in Fig.~\ref{fig:TwoTreesLP}. In the actual implementation, instead of choosing the left-most child and traversing its sub-tree, we first visit the child that gives the shortest error (i.e.\ the one that would have been chosen by Babai's algorithm). Then the sub-tree of this most promising `middle' child is traversed. See Fig.~\ref{fig:TwoTreesPrun} for this tree-traversing strategy. 
 
Further, once we reach the `critical' level $k^*$ determined by the maximal $k$ s.t. $\|\wbvec_k \| > \const \alpha q$ (for some input constant $\const$), we consider only one child for all levels below $k^*$. This additional pruning conforms to the Condition 3 of reasonable pruning (see Def.~\ref{def:ReasonablePruning}): once the Gram-Schmidt vectors are long enough and the solution has `survived' until this level (i.e.\ there exist a path $\cvec$ that contains the coefficients of the solution), we can run the efficient (one child-only) Babai's algorithm. 

Obviously, it makes sense to make the enumeration tree `bushier' on the levels where the $\wbvec_k$'s are relatively short. This is controlled by the function $\B$. In our implementation, it is the linear length pruning function with an additional parameter that controls how wide the tree is allowed to be. 

\subsection{Parallel implementation} \label{sec:MultiThread}

In Alg.~\ref{alg:GenPrunDepth}, sub-tree traversals for two different nodes on the same level are
independent, so we can parallelize the algorithm. Let $\nT$ be the number of
threads (processors) at our disposal. Our goal is to determine the upper-most level $k$ that has
at least as many nodes $\nNodes(k)$ as $\nT$. Then we can traverse the $\nNodes(k)$
sub-trees in parallel by calling Alg.~\ref{alg:GenPrunDepth} on each thread.

We start traversing the enumeration tree in a \emph{breadth-first} manner using a queue.
In a breadth-first traversal, once all the nodes of level $k$ are visited, the queue
contains all their children (i.e.\ all the nodes of level $k+1$), thus their number
$\nNodes(k+1)$ can be computed from the size of the queue. Once a level $k$ with $\nNodes(k) \geq \const \cdot \nT$ for
some constant $\const \geq 1$ is found, we stop the breadth-first traversal and start
Alg.~\ref{alg:GenPrunDepth} for each of the $\nNodes(k)$ sub-trees separately on each thread. The benefit
of having $\const>1$ is that whenever one of the threads finishes quickly, it can be assigned
to traverse another sub-tree. This strategy compensates for imbalanced sizes of sub-trees.

This breadth-first traversal is described in Alg.~\ref{alg:Breadth_first}. At the root we
have $\nNodes(m)=1$. The associated data to each node are the target
$\tvec^{(m-1)}$, the error-length $e^{(m-1)}$, and the partial solution $\xvec^{(m-1)}$.
We store them in queues $Q_t, Q_e, Q_x$. Traversing the tree down is realized via
dequeuing the first element from a queue (line 9) and enqueuing all its children into the
queue. When Alg.~\ref{alg:Breadth_first} terminates, we spawn a thread that receives as
input a target $\tvec^{(k)}$ from $Q_t$, an accumulated so far error-length $e^{(k)}
\in Q_e$, a partial solution $\xvec^{(k-1)} \in  Q_x$, GSO-lengths $(\|\wbvec_{k-1} \|,
\ldots, \|\wbvec_{1} \|)$, and bounding functions $\B^{(i)}$, $1 \leq i \leq k-1$. Since
the number of possible threads is usually a small constant (30-40 on the cluster we are using), there is no blow-up in memory
usage in the breadth-first traversal.

Note that for a family of bounding functions $\B^{(k)}$ that allows to
compute the number of children per node without actually traversing the tree,
e.g.\ the Lindner-Peikert bounding strategy, it is easier to find the level where we start
parallelization. In case of Lindner-Peikert, $\nNodes(k) = \prod_{i=m}^{m-k}
d_i$ and hence, we simply compute the largest level $k$ where $\nNodes(k) \geq \const \cdot \nT$.

In the implemented algorithm we slightly modify the above breadth-first traversal: before starting threads with $\nT$ elements from the queue, we sort the queues $Q_t, Q_e, Q_x$ w.r.t.\ the elements from $Q_e$ s.t.\ the paths with shorter error-length are scheduled first. This might be implemented via priority queues or changing the container type to list and sorting the resulting list. This might speed-up the enumeration if we additionally abort the tree-traversal once we have a leaf with the error of length $\const \cdot  \sqrt{m}\alpha q$ for some input constant $\const$. With this, we exploit the fact that the correct error-vector is much shorter than any other error-vector considered by the algorithm. 


%
% Breadth-First
%
\begin{algorithm}[h]
\caption{Traverse Breadth-First $(\BMat, \protect \tvec, \B^{(k)}, \const)$}
\label{alg:Breadth_first}
\textbf{Input:} $\BMat=(\bvec_1, \ldots, \bvec_m) \in \Z^{m \times m}, \tvec \in \Z^m$, a family of bounding functions $\B^{(k)}$, $\nT \in \Z$, $\const \in \Z$ \\
\textbf{Output:} An array ${(\tvec^{(k)})}_i$ of size $\nNodes(k)$, where $\nNodes(k) \geq \const \cdot \nT$, an array of associated error-length ${(e^{(k)})}_i$, an array of associated partial solutions ${(\xvec^{(k)})}_i$, $1 \leq i \leq \nNodes(k)$.
%\vspace{8pt}
\begin{algorithmic}[1]
\State{} Initialize queues $Q_t, Q_e, Q_x$
\State{} $\Enq{Q_t}{\tvec}$, $\Enq{Q_e}{0}$, $\Enq{Q_x}{\zerovec}$
\State{} Let $\widetilde{\BMat}\gets\GSO(\BMat)$
%\If {$m=0$} \Return $(Q_t, Q_e)$
%\EndIf
\State{} $\nNodes(m) \gets 1$
\State{} $k \gets m-1$
\While{($\nNodes(k+1) < \const \cdot \nT$)}
    \State{} $\nNodes(k) \gets 0$
    \For{$j=1 \ldots \nNodes(k+1)$}
        \State{} $\tvec \gets \Deq{Q_t}$, $e \gets \Deq{Q_e}$, $\xvec \gets \Deq{Q_x}$
        \State{} $\nNodes(k) \gets \nNodes(k)+ \lceil \sqrt{ B^{(m)}(e) } / \| \wbvec_m \| \rceil $
        \State{} $c^{*} \gets \langle \tvec, \wbvec_m  \rangle / \|\wbvec_{m}\|^2$
        \For{$i=0 \ldots \lceil \sqrt{ \B^{(m)}(e) } / \| \wbvec_m \| \rceil  - 1$}
            \State{} $\Enq{Q_t}{\tvec - \lceil c^{*} \pm  i \rfloor \bvec_k}$
            \State{} $\Enq{Q_e}{e+{(c^{*} - \lceil c^{*} \pm  i \rfloor)}^2 \|\wbvec_{k}\|^2}$
            \State{} $\Enq{Q_x}{\xvec + \lceil c^{*} \pm  i \rfloor \bvec_k}$
        \EndFor{}
    \EndFor{}
    \State{} $k \gets k-1$
\EndWhile{}
\State{} \Return{} $(Q_t, Q_e, Q_s)$
\end{algorithmic}
\end{algorithm}

\subsection{Attacks on Variants of \LWE} \label{sec:AttacksOnVariants}

In \cite{STOC:BLPRS13}, the classical hardness of \LWE is proved via a reduction to the so-called binary-\LWE, where the secret vector $\svec$ is chosen from $\{ 0,1\}^n$. While this version of \LWE is shown to be at least as hard as standard \LWE in \cite{STOC:BLPRS13}, the reduction loses a factor of $\log q$ in the dimension. Further, Kirchner and Fouque show in \cite{C:KirFou15} that for binary \LWE, a version of the \BKW algorithm achieves slightly sub-exponential running time of order $\smash[t]{2^{\bigO( \frac{n}{\log \log n} )}}$. The \BDD attack can also profit from the fact that the secret is smaller than the error: Bai and Galbraith describe in \cite{ACISP:BaiGal14} how to tweak a \BDD instance for a faster attack. Our experiments confirm (see Table~\ref{table:RunTimesBinSecret}) that indeed for binary-\LWE we can tackle considerably higher dimensions than for standard \LWE. Further, the dimensions we attack are larger than those solved in \cite{C:KirFou15} yet using much lower sample-complexity.

While binary-\LWE remains to be hard and only requires to slightly increase the lattice-dimension in order to achieve the same security-level as standard \LWE, other modifications to \LWE might be fatal for cryptographic applications. For instance, we show that the cryptosystem based on the hardness of `binary' matrix \LWE proposed in \cite{Galb}, can be broken in several hours for relatively large dimensions using the \BDD attack. We note that we attack Regev's \emph{cryptosystem} (\cite{STOC:Regev05}) instantiated with a binary matrix. The binary matrix \LWE problem itself remains an interesting cryptanalytic target.

\paragraph{Binary secret \LWE.} To speed-up the binary-secret \LWE attack, Bai and Galbraith transform a \BDD instance $(\qLat(\AMat\transpose), \bvec)$ with error $\evec$ into a \BDD instance
\begin{equation} \label{eq:binLWEBDDInstance}
 \left( \qLATTp( \Id_m \, | \, \AMat\transpose), (\bvec, \zerovec) \right)
\end{equation}
with the unbalanced error $(\evec, \svec)$. The instance is correctly defined since
\[
	\footnotesize\arraycolsep=0.8\arraycolsep
	( \Id_m \, | \, \AMat\transpose) \Bigg[ \begin{pmatrix}  \evec \\ \svec \end{pmatrix} - \begin{pmatrix}  \tvec \\ \zerovec \end{pmatrix} \Bigg] = 0 \bmod q.
\]
The lattice $\qLATTp( \Id_m \, | \, \AMat\transpose) \in \Z^{m+n}$ is generated by the columns of $\AMat^{\perp}$ given by
\[
	\AMat^{\perp} = \left(
	\begin{matrix}
	-\AMat\transpose \\
	\const \Id_n \\
	\end{matrix}
	\, \middle\vert \,
	q\Id_{n+m}
	\right),	
\]
where $\const$ is some input parameter that re-balances the error by increasing the determinant of the lattice. Larger determinant and hence, larger $\lambda_1(\AMat^{\perp})$, will speed-up the \BDD attack.

We run the \BDD enumeration algorithm described in Alg.~\ref{alg:GenPrunDepth} in both single- and multi-threaded variants. The results are presented in the table below. Notice that in contrast to the \BKW attack of Kirchner and Fouque \cite{C:KirFou15}, we choose as few samples as possible (while keeping a unique solution) to aid the reduction step. Concretely, we used only $m=150$ \LWE samples as opposed to $2^{28}$ samples required in the \BKW attack. We also observe an almost perfect speed-up during a 10-threaded run on an instance of dimension $140$.  
\vspace{20pt}
\begin{table}[h]
	\centering
	\begin{tabular}{cccc|cc|cc}
		\toprule
		\multicolumn{4}{c|}{\LWE-parameters}               & \multicolumn{2}{c|}{\BKZ-reduction}  & \multicolumn{2}{c}{Length Pruning} \\
		$n$   & $q$      & $\alpha$                     & $m$   & $\beta$ & $T$         & $\nT$ &  $T$      \\\midrule
		$120$ & $16411$  & $0.001$                     & $150$ & $10$    & $2.3$h      &  $1$  & $2$h     \\
		$130$ & $16411$  & $0.001$                     & $150$ & $15$    & $6.6$h      &  $1$  & $1$h     \\ \midrule
		$140$ & $16411$  & $0.001$                     & $170$ & $15$    & $12$h       &  $1$  & $16.3$h  \\
		$140$ & $16411$  & $0.001$                     & $170$ & $15$    & $12$h       & $10$  & $1.7$h   \\\midrule[1pt]
	\end{tabular}
	\caption{Running times of the \BDD-decoding attack on binary secret \LWE}
	\label{table:RunTimesBinSecret}
\end{table}
\vspace{20pt}

\paragraph{Binary matrix.} To implement an \LWE-based encryption on lightweight devices, \cite{Galb} proposed not to store the whole random matrix $\AMat \in \Z_q^{n \times m}$, but to generate the entries of a \emph{binary}
$\AMat \in \Z_2^{n \times m}$ via some Pseudorandom Number Generator. Galbraith's ciphertexts are of the form
\[(C_1, C_2) = (\AMat \uvec, \langle \uvec, \bvec \rangle + m \lceil q/2 \rceil \bmod q)
\]
for a message $m \in \{0,1\}$, some uniformly random $\uvec \in {\{0,1\}}^m$ and a modulus
$q \in \Z$. The task is to recover $\uvec$ given $(\AMat, \AMat \uvec)$.

Let us describe a simple lattice attack on the instance $(\AMat, \AMat \uvec)$. Notice
that $C_1 = \AMat \uvec$ holds over $\Z$ and, hence, over $\Z_q$ for large enough
modulus $q$ since we expect to have $\AMat \uvec \approx m/4 < q$. First, we find any solution
$\wvec$ for $\AMat \wvec = C_1 \mod q$. Note that
\[(\wvec- \uvec) \in \ker (\AMat).
\]
So
we have a \BDD{} instance $(\qLATTp (\AMat), \wvec)$, with $\uvec$ as the
error-vector of expected length $m/2$ and a lattice with $\det(\qLATTp(\AMat))= q^n$.
Since we can freely choose $q$ to be as large as we want, we can guarantee that
$\lambda_1 (\qLATTp(\AMat)) \gg m/2$. Such an instance can be solved by first
running $\beta$-\BKZ{} for some small constant $\beta$ and then Babai's $\CVP$ algorithm.

As a challenge, Galbraith proposes a parameter-set $(n=256, m=400)$ and estimates
that computing $\uvec$ from $\AMat \uvec$ should take around one day.
We solve this instance using NTL's \BKZ{} implementation with $\beta = 4$ and $q
= 500009$ in 4.5 hours.

\vspace{20pt}
\begin{table}[h]
	\centering
	\begin{tabular}{cccc|c|cc}
		\toprule
		\multicolumn{4}{c|}{\LWE-parameters}               & {\BKZ-reduction}  & {Babai's \CVP} \\
		$n$   & $q$          & $m$   & $\beta$ & $T$        &  $T$      \\\midrule
		$256$ & $500009$     & $400$ & $4$     & $4.5$h     & $2 \text{min}$                  \\
		$280$ & $500009$     & $440$ & $4$     & $6.5$h     & $3 \text{min}$                  \\\bottomrule
	\end{tabular}
	\caption{Running times of the \BDD-decoding attack on cryptosystem based on binary matrix \LWE}
	\label{table:RunTimesBinMatrix}
\end{table}

\subsection{Details on Implementation} \label{sec:DetailsOnImplementation}

We implemented our \BDD{} enumeration step choosing Linear Length Pruning as a bounding strategy.
All programs are written in \textsc{C++} and we used \textsc{C++11 STL} for implementing the threading. Our tests were performed on the Ruhr University's
{Crypto Crunching Cluster}~(C3) \cite{C3} which consists of one master node to schedule jobs and four computing
nodes. Each computing node has four AMD Bulldozer Opteron 6276 CPUs, and thus 64~cores,
running at 2.3 GHz and 256 GByte of RAM\@. The results of our experiments are
presented in Table~\ref{tabel:RunTimesLWE}. Let us take a closer look at this table.

All instances are split into tree categories depending on the noise-rate: the left-most have $\alpha=0.001$, middle $\alpha=0.002$, right-most $\alpha=0.005$. The instances for the first two cases were generated by ourselves with modulus $q=4093$, while for the last case, we attack the instances offered by the \LWE-Challenge \cite{LWEChallenge}. For $n=40, 45, 50$, the moduli are $q=1601, 2027, 2053$ respectively.

For the \LWE instances with small error-rate, we took $m=2n$ samples. For $\alpha=0.002$, to aid the enumeration step, we  slightly increase the number of samples to $m \approx 2.2n$. Notice, that for a larger $m$, the determinant of the \LWE-lattice, $\det(\qLat(A\transpose)) = q^{1-n/m}$, increases. This leads to a larger $\lambda_1(\qLat(A\transpose))$, making the error, from the enumeration point of view, closer to the target. Larger $m$ explains why the running-time for $\BKZ$ with the same block-size $\beta$ increases for the large noise-rates. For $\alpha=0.005$, we use even more samples $m \approx 2.3n$. Note that theory suggests an increased $m$ as well: in Thm.~\ref{thm:BalanceSuperExp}, the optimal choice of $m$ was proved to be $m=\left( \frac{2 \cq}{\sqrt{2 \cBKZ}+\ca} + \smallo(1) \right) \cdot n$. Recall that larger noise-rate corresponds to smaller $\ca$. 

For the pruning strategy $\B$, we choose the \emph{Linear-Length} pruning function \cite{EC:GamNguReg10}. This means that our tree-traversal Algorithm \ref{alg:GenPrunDepth} receives on input an $m$-dimensional array $R$ consisting of level-bounds $R_{m-k} = \cf ((m-k) (\alpha q)^2)$, where $k$ goes from $m$ down to $0$. These bounds determine the allowed accumulated error-length per level. $\cf$ is an additional input-constant. The larger $\cf$ we input, the bushier the enumeration tree is and, hence, the more expensive the algorithm is.

Since we know the length of the Gram-Schmidt basis-vectors $\wbvec$, we can determine the maximal level $k$ (called `critical' from our asymptotical analysis in Sect.~\ref{sec:LWEasBDDAs}), for which $\|\wbvec_k \| > \const \alpha q$ for a constant $\const$ which we set $\const = 2$. From this level down, we run Babai's Algorithm~\ref{alg:Babai}.

From the experiments, we draw the following conclusions. 
	
\vspace{8pt} \hspace{5pt} \textbf{Enumeration can be perfectly parallelized.} Indeed, the way we schedule the jobs for the parallel tree-traversal in Algorithm~\ref{alg:Breadth_first} allows for the speed-up equal to the number of available threads. Recall that in Algorithm~\ref{alg:Breadth_first}, we create much more sub-trees (i.e.\ jobs) than the number of threads $\nT$, and store the roots of these sub-trees in a queue. An additional input parameter $\const$ determines the size of this queue. In our tests, we set this parameter large enough to guarantee that the number of jobs in the queue is of order $5000-6000$ (for $\nT = 10$, it corresponds to $\const=500-600$). For the dimensions we tackle, these numbers are larger enough to guarantee that there will many equally big sub-trees and all the threads will be evenly occupied. It is reasonable to predict that for higher dimensions and/or more threads at hand, one should choose queues of larger size. 

An almost perfect speed-up was achieved for all dimensions where we run the parallelized enumeration: for $\alpha=0.001$ and dimensions $n=70,80,90$, our multi-threaded implementation allows to choose relatively small $\beta$'s for the reduction and hence, to balance out the running times for the reduction and enumeration. Based on the experiments for these dimensions, for some other instances only the parallelized version was run. For example, for $n=75$, both reduction and enumeration on 10 threads were finished in about an hour. On instances with larger noise-rates, the tests were mostly run in the multi-threaded regime as enumeration becomes significantly slower.


\vspace{8pt} \hspace{5pt} \textbf{Binary error is significantly easier for enumeration.} We also performed some tests on instances with a binary noise (this version of \LWE also admits a hardness reduction, \cite{C:MicPei13}, but for a restricted number of samples $m = \bigO(n)$). For such a small error-rate, enumeration is fast, so in order to balance the attack, we choose a \emph{smaller} $m$ (but still large enough to guarantee the unique solution). This speeds up the reduction but slows down the enumeration. Again, we mitigate this slow-down with parallelization. In Table~\ref{table:RunTimesBinError} below, we choose $n=130$ as an example. Note that for this dimension, the attack runs in approximately the same time as for $n=75$ in the Gaussian-error case with small noise-rate.  

\vspace{10pt}
\begin{table}[h]
	\centering
	\begin{tabular}{ccc|cc|cc}
		\toprule
		\multicolumn{3}{c|}{\LWE-parameters}  & \multicolumn{2}{c|}{\BKZ-reduction}  & \multicolumn{2}{c}{Length Pruning} \\
		$n$   & $q$                      & $m$   & $\beta$ & $T$         & $\nT$ &  $T$      \\\midrule     
		$130$ & $4093$     & $190$ & $18$     & 1.6e4     & 1 & 4.8e4  \\ 
		$130$ & $4093$     & $190$ & $18$     & 1.6e4     & 10 & 6.1e3  \\
		\bottomrule
	\end{tabular}
	\caption{Running times of the \BDD-decoding attack on binary \LWE}
	\label{table:RunTimesBinError}
\end{table}

\vspace{8pt} \hspace{5pt} \textbf{An increase in the error-rate causes a substantial slow-down for the attack.} Indeed, in case of the large noise-rate of $\alpha=0.005$, the attack performs significantly worse than for a smaller rates. In order to obtain long enough Gram-Schmidt vectors for successful decoding, we have to (1) increase $m$, and (2) increase $\beta$. Both factors result in slower \BKZ weakening the whole attack. 

\vspace{8pt}

We would like to mention that a couple of months after the publication of \cite{ACNS:KMW16}, the \LWE Challenge was announced in \cite{LWEChallenge}. Currently, the attack that tackles the hardest parameter-sets is the parallelized two-phase decoding.

\clearpage

\thispagestyle{empty} %no page number
\begin{landscape}
\begin{table}
	\centering
	\begin{tabular}{|c|c|c|c|c|c|c|c|c|c|c|c||c|}
	
	\hline 
	\diagbox{$n$}{$\alpha$} & \multicolumn{4}{c|}{$0.001$} & \multicolumn{4}{c|}{$0.002$} & \multicolumn{4}{c|}{$0.005$} \\ \hline
	& $\beta$ & $T(\BKZ)$ & $\nT$ & $T(\ENUM)$ & $\beta$ & $T(\BKZ)$ & $\nT$ & $T(\ENUM)$ & $\beta$ & $T(\BKZ)$ & $\nT$ & $T(\ENUM)$ \\ \hline
	40 & 2 & 1.1e2 & 1 & 5.0e1 & 
	     10& 2.0e2 & 1 & 2.2e2 & 
	     16& 9.1e2 & 10 & 3.02e2 \\ \hline
	45 & 3 & 1.2e2& 1 & 5.1e1 &
	     12 & 2.3e2& 1& 4.5e2 & 
	     19 & 3.2e3& 10 & 3.5e3 \\ \hline 
	50 & 3 & 1.25e2& 1 & 7.3e1& 
	     15& 6.7e2& 5 & 7.6e2& 
	     21& 1.6e4& 10& 2.3e4\\ \hline 
	55 & 5 & 1.5e2& 1 & 1.4e2& 
	     17& 1.7e3& 10 & 2.1e3& & & & \\ \hline 
	60 & 10 & 1.8e2& 1 & 3.1e2&
	     18 & 2.4e3 & 10& 4.0e3& & & & \\ \hline 
	65 & 12 & 2.1e2& 5 & 5.3e2&
	     22 & 9.1e3& 10& 7.8e3& & & & \\ \hline 
	\multirow{2}{*}{70} &\multirow{2}{*}{15} & \multirow{2}{*}{2.5e2} & 1 & 5.04e4& 
	\multirow{2}{*}{25} & \multirow{2}{*}{1.3e4}& 10 & 2.4e4 & & & & \\ \cline{4-5}\cline{8-13} 
	& & & 10 & 5.4e3 & 
	& &   20 & 1.4e4 & & & & \\ \hline 
	75 & 18 & 3.1e3 & 10 & 3.4e3 & & & & & & & & \\ \hline 
	\multirow{2}{*}{80} & \multirow{2}{*}{25} & \multirow{2}{*}{1.5e4} & 1 & 4.68e4& & & & & & & & \\ \cline{4-13} 
	& & & 10 & 5.4e3& & & & & & & & \\ \hline 
	85 & 23 & 1.55e5 & 10 & 2.1e5& & & & & & & & \\ \hline 
	\multirow{2}{*}{90} & \multirow{2}{*}{22} & \multirow{2}{*}{4.07e4} & 1 & 1.28e5& & & & & & & & \\ \cline{4-13} 
	& & & 10 & 1.3e4& & & & & & & & \\ \hline
	 


\end{tabular}
		\caption[Running times of the \BDD-decoding attack on standard \LWE parameters]{Running times (in seconds) of the enumeration attack (Alg.~\ref{alg:GenPrunDepth}) with \emph{Linear-Length} Pruning on \LWE with parameters $(n, \alpha, q=4093)$ for $\alpha=0.001, 0.002$, $(n=40, q=1601), (n=45, q=2027, n=50), q=2503)$ for $\alpha=0.005$.}
		\label{tabel:RunTimesLWE}
\end{table}
\end{landscape}