Abstract

This thesis is dedicated to the analysis of hard problems based on lattices in $n$-dimensional Euclidean space. We consider the problems most relevant to cryptography, the Learning with Errors (LWE) Problem and the Shortest Vector Problem. 

Complexity of Learning with Errors problem, asymptotical and practical, is studied in the first part of the thesis. We analyze the hardness of LWE under lattice-based attacks and provide their running times of the form $2^{\const n + \smallo(n)}$, where we make the constant $\const$ explicit as a function of LWE parameters. From a theoretical perspective, our analysis reveals how the complexity of the problem changes as a function of its parameters. From a practical perspective our analysis is a useful tool to choose LWE parameters resistant to known attacks. We make our study complete by providing real running times of lattice-based LWE solvers for various ranges of parameters.

The second part of the thesis deals with heuristic sieving algorithms for the Shortest Vector Problem. The main result is an algorithm with $2^{0.1877n+\smallo(n)}$ memory-complexity and $2^{0.396n+\smallo(n)}$ running time. The speed-up is obtain from an efficient algorithm for finding a triple of lattice-vectors whose sum has a short Euclidean norm.

\hspace{2cm}

Zusammenfassung

Diese Arbeit besch{\"a}ftigt sich mit der Analyse von harten gitterbasierten Problemen. Wir betrachten Gittern des $n$-dimensionalen Euklidischen Raums. Wir untersuchen zwei Krypto-relevante Probleme: Das Learning with Errors Problem (LWE) und das K{\"u}rzeste Vektor Problem.

Die Komplexit{\"a}t von LWE wird im ersten Teil der Arbeit betrachtet. Wir untersuchen alle gitterbasierten Angriffe auf LWE. Die Laufzeiten dieser Angriffe werden in der Form $2^{\const n + \smallo(n)}$ angegeben, wobei die Konstante $\const$ explizit als die Funktion der LWE-Parameter bestimmt wird. Aus unserer Analyse kann man ableiten wie sich die Komplexit{\"a}t des Problems bei verschiedenen LWE Parametern ver{\"a}ndert. Dies ist hilfreich, wenn man LWE Parameter auw{\" a}hlen will, die resistent gegen allen bekannten gitterbasierten Angriffen sind. Au{\ss}erdem, geben wir die Laufzeiten der gitterbasierten Angriffe mit verschiedenen LWE-Parametern an, die mithilfe unseres LWE-Solvers best{\"a}tigt wurden.

Der zweite Teil der Arbeit besch{\"a}ftigt sich mit sogenannten Sieving Algorithmen f{\" u}r das K{\"u}rzeste Vektor Problem. Wir stellen einen neuen Algorithmus vor, der die Laufzeit $2^{0.396n+\smallo(n)}$ und die Speicherkomplexit{\"a}t $2^{0.1877n+\smallo(n)}$ hat. Die Verbesserung basiert auf einem effizienten Algorithmus, der drei Gittervektoren findet, so dass die Summe dieser drei Vektoren ein kurzer Vektor ist.
