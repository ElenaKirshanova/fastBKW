First and foremost, my deepest thanks go to my advisor Alexander May. I feel privileged to have his guidances and support over the last three years. I would like to thank him for the immense amount of time he spent explaining me what cryptography is actually about and for his endless patience when I was slow on the uptake. I was extremely fortunate to have such an amazing advisor.

Much of this Thesis is joint work with Gottfried Herold, to whom I am much indebted for his vital contribution to my understanding of math and crypto. Not only is he a person able to elegantly solve math problems when I got stuck, but also a good friend. I am grateful to my other co-author, Friedrich Wiemer, for settling out my countless questions about programming.

I thank the whole crypto group at RUB for the invaluable support and encouragement especially during the last and toughest months of my PhD. Being a part of such a group is a privilege. A special thanks go to my office mate, Felix Heuer, not only for proof-reading my entire Thesis within one day, but also for making our office NA 5/75 a place full of joy and fun, not a place full of PCB. Marion Reinhardt-Kalender, whose help and assistance made my stay in Germany most comfortable and untroubled, deserves a special \textit{Vielen Dank}.

During my last year, I had a great honour to collaborate with the crypto group in ENS Lyon lead by Damien Stehl{\'e}. I have learned a great deal during my visit there and I am looking forward to working together.

I would not have even considered doing research in crypto, if I had not been introduced to the subject during my studies at I.Kant Baltic Federal University in Kaliningrad. I would like to thank S. Aleschnikov, A. Zaytzev, and all the others members of the Faculty of Mathematics at BFU for their inspiring lectures.


Most importantly, I would like to thank my family and especially my mother, for her constant support and her firm belief in me. 





