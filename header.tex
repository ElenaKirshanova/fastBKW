% !TEX root = main_spring.tex
% ==================================================================
% Definitions for this paper
% ==================================================================

\mathchardef\hyphen="2D

% ADVERSARIES AND SUCH
\newcommand*{\poly}{\ensuremath{\mathrm{poly}}}
\newcommand*{\eps}{\ensuremath{\varepsilon}}

% GROUPS/DISTRIBUTIONS/SETS/LISTS
\newcommand{\N}{{{\mathbb N}}}
\newcommand{\Z}{{{\mathbb Z}}}
\newcommand{\Q}{{{\mathbb Q}}}
\newcommand{\R}{{{\mathbb R}}}
\newcommand{\F}{{{\mathbb F}}}
\newcommand{\CC}{{{\mathbb C}}}
\newcommand*{\IZ}{\ensuremath{\mathbb{Z}}}
\newcommand*{\IN}{\ensuremath{\mathbb{N}}}
\newcommand*{\IR}{{{\mathbb R}}}
\newcommand{\Zp}{\ints_p} % Integers modulo p
\newcommand{\Zq}{\ints_q} % Integers modulo q
\newcommand{\Zn}{\ints_N} % Integers modulo N
\newcommand{\Zr}{\ensuremath{\mathbb{Z}/r\mathbb{Z}}} % Integers modulo N
\newcommand*{\dDR}{\mathrm{d}} %de-Rham-Differential (the d in dx, dy, dz and so on)
\newcommand{\transpose}{\mkern0.1mu^{\mathsf{t}}}
\newcommand*{\union}{\mathbin{\cup}}
\newcommand{\HNF}{\mkern0.1mu^{\scalebox{0.75}{\texttt{HNF}}}}




\newcommand{\bigO}{\mathcal{O}}
\newcommand*{\OLandau}{\bigO}
\newcommand*{\WLandau}{\Omega}
\newcommand*{\xOLandau}{\widetilde{\OLandau}}
\newcommand*{\xWLandau}{\widetilde{\WLandau}}
\newcommand*{\TLandau}{\Theta}
\newcommand*{\xTLandau}{\widetilde{\TLandau}}
\newcommand{\smallo}{o} %technically, an omicron
\newcommand{\softO}{\widetilde{\bigO}}
\newcommand{\softTheta}{\widetilde{\Theta}}
\newcommand{\wLandau}{\omega}
\newcommand{\negl}{\mathrm{negl}}
%\newcommand{\poly}{\mathrm{poly}}


% Lattices
\newcommand*{\LLL}{\ensuremath{\mathtt{L}^{\mkern-3mu 3}}}
\newcommand{\Lat}{\mathcal{L}} %lattice generated as image (rather than kernel).
\newcommand{\LATTp}{\mathcal{L}^{\perp}}
\newcommand{\dual}{\mkern0.1mu^{\star}}
% \newcommand{\coset}{\Lambda} % Lambda Lattice
% \newcommand{\cosetPerp}{\Lambda^{\bot}} % Lambda_Perp Lattice

% PROBABILITY SYMBOLS
\newcommand*\PROB\Pr 
\DeclareMathOperator*{\EXPECT}{\mathbb{E}}
\newcommand{\Uniform}{\ensuremath{\mathcal{U}}}
\DeclareMathOperator*{\Tr}{\ensuremath{Tr}}

% VECTORS AND MATRICES
\newcommand{\vc}[1]{\ensuremath{\mathbf{#1}}}%
\newcommand{\bvec}{\ensuremath{\mathbf{b}}} 
\newcommand{\cvec}{\ensuremath{\mathbf{c}}}
\newcommand{\svec}{\ensuremath{\mathbf{s}}}
\newcommand{\uvec}{\ensuremath{\mathbf{u}}}
\newcommand{\vvec}{\ensuremath{\mathbf{v}}}
\newcommand{\wvec}{\ensuremath{\mathbf{w}}}
\newcommand{\yvec}{\ensuremath{\mathbf{y}}}
\newcommand{\xvec}{\ensuremath{\mathbf{x}}}
\newcommand{\zvec}{\ensuremath{\mathbf{z}}}
\newcommand{\evec}{\ensuremath{\mathbf{e}}}
\newcommand{\onevec}{\ensuremath{\mathbf{1}}}
\newcommand{\zerovec}{\ensuremath{\mathbf{0}}}
\newcommand{\deltavec}{\ensuremath{\mathbf{\delta}}}
\DeclareMathOperator{\Ker}{Ker}
\DeclareMathOperator{\Image}{Im}
\newcommand{\qcube}{\ensuremath{\llbracket -\frac{q}{2}, \frac{q}{2} \llbracket}}


% \newcommand{\AMat}{\textbf{A}} %A matrices
% \newcommand{\BMat}{\textbf{B}} %B matrices
% \newcommand{\RMat}{\textbf{R}} %R matrices
% \newcommand{\HMat}{\textbf{H}} %H matrices
 \newcommand{\XMat}{\textbf{X}} %X matrices
% \newcommand{\mbar}{\bar{m}} %mBar dimension
% % \newcommand{\gauss}{\mathcal{D}} % gaussian distribution
 \newcommand{\Id}{\mathbb{I}} % Identity matrix
% \newcommand{\ZeroM}{\textbf{0}} % zero matrix
% \newcommand{\er}{\textbf{e}} % gaussian distr. vectors
% % \newcommand{\cipher}{\textit{c}} % ciphertext
% \newcommand{\Olwe}{\mathcal{O}_{\textsf{LWE}}} %LWE oracle
% \newcommand{\OSample}{\mathcal{O}_{Sample}} %LWE oracle
% \newcommand{\SigmaB}{\boldsymbol{\Sigma}} %semi-deifinite matrix Sigma%
% % \newcommand{\mods}{\text{ mod}}



% Really wide hat (e.g., for Fourier transform)
\newcommand\reallywidehat[1]{\arraycolsep=0pt\relax%
	\begin{array}{c}
		\stretchto{
			\scaleto{
				\scalerel*[\widthof{\ensuremath{#1}}]{\kern-.5pt\bigwedge\kern-.5pt}
				{\rule[-\textheight/2]{1ex}{\textheight}} %WIDTH-LIMITED BIG WEDGE
			}{\textheight} % 
		}{0.5ex}\\           % THIS SQUEEZES THE WEDGE TO 0.5ex HEIGHT
		#1\\                 % THIS STACKS THE WEDGE ATOP THE ARGUMENT
		\rule{-1ex}{0ex}
	\end{array}
}


%argmin, argmax
\DeclareMathOperator*{\argmin}{\arg\!\min}
\DeclareMathOperator*{\argmax}{\arg\!\max}


%Norms and Scalar products
\newcommand*\abs[1]{\left\lvert#1\right\rvert}
\newcommand*\norm[1]{\left\lVert#1\right\rVert}
\newcommand*\normalabs[1]{\lvert#1\rvert} 
\newcommand*\normalnorm[1]{\lVert#1\rVert}
\newcommand*\bignorm[1]{\bigl\lVert#1\bigr\rVert}
\newcommand*\bigabs[1]{\bigl\lvert#1\bigr\rvert}
\newcommand*\Bigabs[1]{\Bigl\lvert#1\Bigr\rvert}
\newcommand*{\ScProd}[2]{\ensuremath{\langle#1\mathbin{,}#2\rangle}} %Scalar Product
% \newcommand*{\ScProd}[2]{\ensuremath{\langle#1 \:{,}\:#2\rangle}} %Scalar Product
\newcommand*{\bigScProd}[2]{\ensuremath{\bigl\langle#1\mathbin{,}#2\bigr\rangle}} %Scalar Product
\newcommand*{\BigScProd}[2]{\ensuremath{\Bigl\langle#1\mathbin{,}#2\Bigr\rangle}} %Scalar Product


%Some other math operators

\DeclareMathOperator{\Span}{Span} %span of vectors
\DeclareMathOperator{\vol}{\mathrm{vol}} %volume

\newcommand{\minus}{\scalebox{0.75}[1.0]{$-$}}
\newcommand{\overbar}[1]{\mkern 1.5mu\overline{\mkern-1.5mu#1\mkern-1.5mu}\mkern 1.5mu}

\DeclareMathOperator{\erf}{erf} %error function
\DeclareMathOperator{\erfc}{erfc} %complementary error function
\newcommand{\Er}{\ensuremath{\mathrm{Er}}} %complementary error function


%Thick line for table
\setlength{\doublerulesep}{0pt}
\newcommand{\thickline}{\hline\hline\hline}


%circled text
\newcommand*\circled[1]{\tikz[baseline=(char.base)]{
    \node[shape=circle,draw,inner sep=0.3 pt] (char) {\scriptsize #1};}}


%Fix Algorithmicx package
\def\NoNumber#1{{\def\alglinenumber##1{}\State #1}\addtocounter{ALG@line}{-1}}


%For comments within algorithms
\algrenewcommand\algorithmiccomment[2][\normalsize]{{#1\hfill\(\triangleright\) #2}}

%ordinal numbers
\renewcommand{\th}{^{\mathrm{th}}}
\newcommand{\st}{^{\mathrm{st}}}
\newcommand{\nd}{^{\mathrm{nd}}}


% To make a footnote without the number
\newcommand\blfootnote[1]{%
	\begingroup
	\renewcommand\thefootnote{}\footnote{#1}%
	\addtocounter{footnote}{-1}%
	\endgroup
}
%Proper limit with the subscript underneath
% \newcommand{\Lim}[1]{\raisebox{0.5ex}{\scalebox{0.8}{$\displaystyle \lim_{#1}\;$}}}